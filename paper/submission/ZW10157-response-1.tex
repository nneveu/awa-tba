\documentclass{article}
\usepackage{latexsym}
\usepackage{color}
\setlength{\parskip}{3mm}
\setlength{\parindent}{0pt}

\newcommand{\TODO}[1]{\color{green}\noindent[{\bf TODO: #1}]\color{black}}
\newcommand{\atline}[1]{\color{red} Lines: #1\color{blue}}
\begin{document}
\title {Paper  \#ZW10157}
\author {N. Neveu, et. al}

\maketitle
Dear reviewers and editors,

Thank you for considering the manuscript.
We appreciate the constructive criticism. It helped improve the quality of this manuscript.

We have carefully answered all your questions and implemented your suggestions to the best of our knowledge.\ The details can be found in the following pages.

With the best regards  \\[3mm]
 
Nicole Neveu, et. al

\pagebreak

{\bf Problems with the manuscript:}

Figure(s) [10,11]
Please remove background grid and/or shading from your figures.

* Please provide a figure file (.ps, .eps, .pdf, .jpg, or .png format)
that does not require tikz commands in the text source file to
display properly. Figure files should be self-contained and we do
not support the tikz format in the production process. We will also
need a revised text source file.

{\bf My response:} {\color{blue} The grid has been removed from Figures 10 and 11.
PDF versions of the tikz pictures have been made. All updated figures are provided 
along with an updated source file.}


{\bf Referee A (Remarks to Author)}

1) Remove the sections of code from the text. This adds no value to
the paper and should be left for a User’s Manual.

{\bf My response:} {\color{blue} The code sections are now removed.}


2) Provide a stronger physics basis for the AWA optimization. Why is
this problem of beam squeeze challenging enough to warrant solution
with such a massively capable solver? Could the same solution have
been arrived at with something simpler? Describe the important
component of physics that come into play in the beam squeeze, whatever
they are – space charge, etc. After all, this is primarily a physics
journal so the physics context for the application needs to be clear.

{\bf My response:} {\color{blue} XXXXX.}



REFEREE B - ZW10157

First, optimization systems based on PISA have been in use in
accelerator design since 2005, and NSGA-II’s use dates from that time
too. References to early PISA-based systems and NSGAII applications
are missing from the authors’ reference list, and the authors must
correct this. Three groups of references are provided to assist in
addressing these omissions: 1) early PISAbased systems; 2) a small
sample of NSGA-II applications and implementations (there are MANY
more as the NSGA-II algorithm is easy to understand and fairly
straight forward to implement); and 3) a survey of accelerator physics
and genetic and evolutionary application usage as of 2013.

1) References for 4 PISA-based optimization systems:

I.V. Bazarov and C.K. Sinclair. "Multivariate optimization of a high
brightness dc gun photoinjector," Phys. Rev. ST Accel. Beams 8,
034202, 2005

Y.C. Chao et al. "TRIUMF-VECC Electron Linac Beam Dynamics
Optimization," ICAP09, San Francisco, 2009, THPSC012

A. Hofler et al. "Optimizing SRF Gun Cavity Profiles in a Genetic
Algorithm Framework," ICAP09, San Francisco, 2009, THPSC020

M. Marchetto et al. "Beam Dynamics Optimization of the TRIUMF ELINAC
Injector," PAC09, Vancouver, BC, Canada, 2009, TH6PFP097

A. Hofler and P. Evtushenko. “Optimizing RF Gun Cavity Geometry within
an Automated Injector Design System,” NAPAC11, New York, 2011, TUODS6

I.V. Bazarov et al., "Comparison of dc and superconducting rf
photoemission guns for high brightness high average current beam
production", Phys. Rev. ST Accel. Beams 14, 072001, 2011

A. Hofler, “Optimization Framework for a Radio Frequency Gun Based
Injector,” Ph.D. thesis, Old Dominion University, Norfolk, Virginia,
USA, 2012

C. Gong and Y.C. Chao, "The TRIUMF Optimization Platform and
Application to the E-LINAC Injector", ICAP2012, Rostock-Warnemunde,
Germany, 2012, TUABI1

A. Hofler et al., "Innovative applications of genetic algorithms to
problems in accelerator physics", Phys. Rev. ST Accel. Beams 16,
010101, 2013

B. Terzic et al., "Simultaneous optimization of the cavity heat load
and trip rates in linacs using a genetic algorithm", Phys. Rev. ST
Accel. Beams 17, 101003, 2014

C. Gong, A Novel Optimization Platform and Its Applications to the
TRIUMF Energy Recovery Linac, Ph.D. thesis, The University of
Columbia, October 2015

C. Gong and Y.C. Chao, "The TRIUMF Optimization Platform and
Application to the E-LINAC Injector", ICAP2015, Shanghai, China, 2015,
THCWC4

C. Gulliford et al., "Multiobjective optimizations of a novel
cryocooled dc gun based ultrafast electron diffraction beam line",
Phys. Rev. Accel. Beams 19, 093402, 2016

C. Gulliford et al., "Multiobjective optimization design of an rf gun
based electron diffraction beam line", Phys. Rev. Accel. Beams 20,
033401, 2017

2) Optimization systems using NSGA-II (a few example references):

L. Emery. PAC 2005, Knoxville, TN, USA, May 2005, RPPP047

M. Borland et al. PAC 2009, Vancouver, BC, Canada, May 2009, TH6PFP062

Pulsar Physics. General Particle Tracer, http://www.pulsar.nl/gpt

3) Survey with various references showing GA/EA usage in Accelerator
Physics: A. Hofler. “Genetic Algorithms and Their Applications in
Accelerator Physics,” NAPAC2013, Pasadena, CA, USA, 2013, THTB1

{\bf My response:} {\color{blue} XXXXX.}





Second, due to the thoroughness and success of Bazarov and Sinclair’s
optimization (PRSTAB 2005 provided above), genetic and evolutionary
algorithm-based optimizations are standard practice in accelerator
design, and researchers are now looking at ways to use machine
learning and AI techniques in accelerator design and optimization.
Manual parameter scans, as mentioned in Section I lines 17-19, are
useful, but they are by no means the sole optimization technique in
use today. The statement made in lines 17-19 was true prior to 2005
but not now. The authors should consider providing a different and
more compelling motivator to encourage users to consider the system
they have developed.

{\bf My response:} {\color{blue} XXXXX.}

Third, several of the features of this system are available is
existing systems but are not acknowledged:

- parallel/distributed computation: the survey article mentioned above
provides references to several examples

- Section II lines 115-125 blurring generations: APISA and General
Particle Tracer’s optimization system start evaluating the next
generation before all individuals in the present generation complete,
too.

- Section III subsection D lines 229-231 Geometry optimization. With
PISA-based systems, Bazarov and Hofler have done this (see PISA-based
reference list). Others have performed geometry optimization with
different GA/EA based systems (again the survey reference mentions
several.)

{\bf My response:} {\color{blue} XXXXX.}

That said, a general-purpose optimization system is a useful addition
to the design tools for accelerators. The way to entice prospective
users through PRAB is to introduce the system with a very brief
general overview and then focus on the results obtained using the
system. The detailed discussion of the design and implementation with
code snippets should be provided as Supplemental Material. PRAB is not
a computer journal. Salient implementation or design points if
necessary can be summarized in short Appendices; the main discussion
of the paper should be the optimization results not the details of the
implementation and design. For example, in the main text, it is
reasonable to say that the optimization system provides a flexible and
extensible meta-language for setting up optimization problems (covered
in detail in Section III subsection E) and then direct the reader to
an Appendix or the Supplemental Material for the details provided in
Section III subsection E. Bear in mind, though, that several of the
PISA-based systems mentioned above, also, provide expression
capabilities and should be referenced. The one presented here is more
flexible and has the advantage of being easily extended. Both are
distinguishing features. The authors need to emphasize what
differentiates their system from 13+ years of development in the area
of GA/EA-based optimization in accelerator design and then demonstrate
that the results produced are exemplary or could not have been
produced otherwise. Or they should focus on how their system advances
the application of the target AWA machine (the Gulliford references
use this approach).

Following are comments/suggestions/questions organized by paper
section.

Abstract:

In this paper, we present the implementation of such a general-purpose
framework for simulation-based multi-objective optimization methods
that allows the automatic investigation of optimal sets of machine
parameters. The implementation is based on a master/slave paradigm,
employing several masters that govern a set of slaves executing
simulations and performing optimization tasks. -> In this paper, we
present a general-purpose framework for simulation-based
multi-objective optimization methods that allows the automatic
investigation of optimal sets of machine parameters.

The high charge beam line at the Argonne Wakefield Accelerator
Facility was used as the beam dynamics model. The 3D beam size,
transverse momentum, and energy spread were optimized. Please state
the significance of the optimization. Is the solution found better
than the original design? Then provide a comparison with the baseline
solution as Bazarov and Sinclair did in their 2005 paper. Is it the
first such solution for the AWA? Then make that clear. Has this
solution been used in the machine? Then provide a comparison between
prediction and operational experience.

Section I

line 35: trade of -> trade off

{\bf My response:} {\color{blue} This typo is corrected in the updated manuscript.}

Section II

lines 53-74. The car example is from Deb’s book but not cited. This
example can probably be removed as EA/GA methodology is established
practice now.

Section III

subsection A

Needs to be updated to reflect developments in accelerator physics.

subsections B-F

Replace with a brief overview and put the details in Supplemental
Material

Section VI

Code and input file details should be in Supplemental Material

Section V

subsection B

Code and input file details should be in Supplemental Material

subsubsection 3

line 427 to to -> to

{\bf My response:} {\color{blue} This typo is corrected in the updated manuscript.}

subsubsection 4

FIGS. 11/12 The discussed case in FIG. 12 is not on either front in
FIG. 11. Explain why a nonfront solution is used otherwise the
optimization exercise is moot.

TABLE IV is provided and not discussed. Are these the settings for the
solution in FIG. 12? The table’s title is confusing. If the table
contents are input parameters for an optimization, shouldn’t they have
ranges of values? If the values are a selected solution, shouldn’t the
title state that directly?

{\it Your manuscript, "On Uncertainty Quantification in Particle Accelerator Modelling," has completed this round of review. The AE and first two reviewers are positive about the main concept of the article - UQ for charged particle accelerators - but have issues with the paper. The manuscript is a narrow in its focus on UQ and the tie to particle accelerator applications is difficult to follow. 

The AE's suggestion to better motivate the use UQ methods is a good one. I like the suggestion of the AE and reviewer 2 to focus on a particular application of using a charged particle accelerator, describing the details so that a general audience can understand, and clearly motivating the need for UQ. This will give you the opportunity to show how UQ makes the science better. If the application is very compelling, there is not a need to be using novel UQ methodology. In this case, it's the application that is novel. A genuine example/application, involving real data, would help help make the case for the need for UQ in this area. }

{\bf My response: }{\color{blue} In the revised version I added a contribution to an ongoing design effort (DAE$\delta$ALUS/IsoDAR) where UQ was used. For this the new section 5 is introduced and the section 4 is renamed to {\em APPLICATION OF THE UQ FRAMEWORK to a MODEL PROBLEM}. I also elaborate more in detail why the 
particular selection of a Cyclotron, is a very general example of a particle accelerator (\atline{281-299}), that justifies the general title and, hopefully refute the claim, that the paper has a  narrow scope.}


{\bf Associate Editor (Remarks to the Author): }

{\it The authors appear to have the goal of providing a broad paper illustrating issues pertaining to uncertainty quantification to the community of scientists modeling particle accelerators. This is a laudatory and timely goal.
Unfortunately, by trying to achieve this breadth, the authors provided a discussion of surrogate models and sensitivity analysis that is fairly standard in the UQ community without adequately demonstrating its novelty for particle accelerator modeling or more general applications. The first and second reviewers both note that the necessity for computing a PCE surrogate, which can be used for subsequent sensitivity analysis, needs to be better motivated for this application. Moreover, statements that the framework provides a template for other computationally intensive applications need to be mollified or better motivated. Both reviewers also note that the discussion regarding extrapolation needs to be better motivated.}

{\bf My response: }{\color{blue}  In section 3 (\atline{281-299}), I added a  discussion of the general Hamiltonian that is the base of the selected Cyclotron as an example.\ This will better motivate the choice of the example.}


{\it The second reviewer discusses two possible directions in which a revision might be considered. The first would be to more adequately introduce UQ to the community of particle accelerator models. The second would be to focus the scope while addressing the concerns raised by all three reviewers. The latter is closer to the present scope of the paper. }

{\bf My response: }{\color{blue} I modify the paper according the second direction} 

{\it If the authors choose to revise the paper by providing a more focused case study, I would encourage them to devote more details illustrating the techniques for a specific application where they can more completely relate the mathematics and statistics to relevant physics. }

{\bf My response:} {\color{blue}I indeed add a more focused case study in section 5.}


{\bf Referee \#1 (Remarks to the Author): }

{\it The work presented in the manuscript lacks a development or findings that have impact on either theory of UQ and its application. Discussion of the UQ approach and associated methods is not new, and numerical results presented are of demonstration value and well expected. The technical content boils down to construction of PC-based surrogate models. In demonstrating the surrogate construction approach, it is important to establish a range of applicability (e.g., depending on physics model properties/ nonlinearity, dependency of parameters). 

The "claim to present a problem that can be recognised as a template for many high intensity modelling attempts, and beyond" is poorly substantiated. The demonstration is performed for a fairly limited case, with three uncertain independent parameters. Furthermore, the physics problem is largely treated as black box. 
}

{\bf My response: }{\color{blue} I motived the {\it general nature} of the presented model problem at \atline{281-299}. I did not elaborate more on the physics of the model problem, but hope to achieve a better understanding of the physics problem by 
introducing the new section 5.}


{\it  The first figure (Fig 1) is perhaps the most important one in the manuscript, as it points to a (potential) novelty of the work presented; (in option 2) when the high fidelity model u* in D* is used to construct surrogate models. 

The last figure (Fig 12) shows the numerical result, that the surrogate models predict several "extrapolation" points (it is unclear are those outside the high-fidelity domain D*). Unfortunately, the figures are either too general (Fig 1) or implicit (Fig 12) for a reader to relate the two, and appreciate how far is the "extrapolation". If the physics remains in the same regime, there is no surprise that the surrogate models can capture the "extrapolation". For example, it would be useful to provide a clear demonstration of performance and limitations of the surrogate model in the domain beyond D*. }

{\bf My response: }{\color{blue}  Also due to other comments, I will not use the term "extrapolation" but instead I use "prediction". I also partially rewrite 2.1 to accommodate your suggestions \atline{107-131}. \\


Option 1: the red points in figure 12 would correspond to x in Fig 1, hence a point outside the training set and found to be a desired/optimal working point. With the surrogate model
we are doing (work in progress) multi objective optimisation in order to find such points.

Option 2: In a similar way I map out D* with the help of the cheep surrogate model, such that the D* is sufficiently small to be accessible with expensive high fidelity simulations. 

In case the surrogate model is sufficiently close to the high fidelity model, option 2 does need not to be considered. In the new section 5, figure 17 is showing such a case. 



} 



{\bf Referee \#2 (Remarks to the Author): }

{\it Overall Summary: 
This paper describes 
* how a response surface can be constructed using polynomial chaos, 
* how sensitivity (Sobol') inidicies can be computed from this, and 
* a stylized example that applies these methods to analize a cyclotron 
model.

While the title is broad, the actual focus of the paper is pretty narrow, 
focusing on PCE-based sensitivity analysis and response surface 
building. Also, the details, purpose, and motivation of the sylized 
example are not brought out very well. I can envision a paper that would 
be important for introducing UQ to the particle accelerator community 
eventually coming from this manuscript, but it's a long way off now. 
}

{\bf My response: }{\color{blue} I motived the {\it general nature} of the presented model problem at \atline{281-299}\ and added section 5, where this Ansatz is
used in an ongoing physics design effort. }

{\it

A successful revision of this paper would have to go into one of two 
directions. It could either try to be a paper that a) intoduces UQ to this 
community of particle accerator modelers, or (b) it could be a focused case 
study, with some note about how such approaches could be used in other 
particle accelerator applicatios. I think the paper is much closer to (b) 
in its current form. A general paper would be quite an undertaking since 
it would have to survey potential problems in accelerator science as well 
as survey UQ methods for sensitivity analysis, inverse problems, optimal 
design / control, etc. If (b) is the appropriate route, then a real 
example would be great. }

{\bf My response: }{\color{blue} I follow direction b) and added a real example in section 5. }

Comments: 

{\it I like the general idea of demonstrating the utility of modern UQ 
methods for the application area of charged particle accelerators. I think 
such a paper could be very influential. However I think the paper has 
some key issues that need to be corrected: 

The details of the stylized problem are not well laid out. Because of this 
I'm not very clear on why a response surface is being generated, on why 
sensitivity inidicies are being calculated, and what advantage was gained 
by the effort described in the paper. I'm sure there is an advantage; it's 
just not brought out effectively. For example, why are the controllable 
parameters assigned independent uniform distributions? I would have 
expected that you'd want to set these parameters so that the resulting 
beam properties are good for some future experiment or data collection. 
As it stands now, I can't follow the goal of this example. 
}

{\bf My response: }{\color{blue} 

I added in section 5 an example of a real design study that should better motivate the gain 
by the effort described. Just before 3.1 \atline{300-310},  I also address more carefully  the 
benefits of surrogate models and sensitivity analysis.

We use uniform distribution to construct the response surface in order to best 
represent the variability of the selected physics parameters. All of them can be
varied in the real accelerator according to an uniform distribution.

} 

{\it 
The UQ methodology focuses solely on the (non-intrusive) PCE approach 
to emulate the code output (QoI's) and to produce sensitivity indicies. 
So it seems to focus on a specific approach to propagate uncertainty from 
inputs to outputs. That's fine, but the title "UQ in particle accelerator 
modeling" implies something much broader. A title that indicates this focus 
would be more appropriate. }

{\bf My response: }{\color{blue} I agree the promises in reading the title is too much, the title is adjusted to {\bf On Non-Intrusive Uncertainty Quantification and Surrogate Model Construction in Particle Accelerator Modelling}}


{\it 
Also, a more focused paper could essentially 
be a case study, so that the general introductions to UQ and particle 
accelerator modeling could be altered to focus on a specific problem 
where the goals of the effort are more concrete. If you go down this 
path, then an actual example would be great. 
}

{\bf My response: }{\color{blue} I added a real example in the new section 5} 

{\it 
If you want to make this a more general paper as the title suggests, 
then a much broader survey of problems and UQ methods will be needed. 
For example, are there interesting inverse problems or control 
problems in this field that UQ could be applied to? Could such features 
be brought out in a stylized problem? 
}

{\bf My response: }{\color{blue} At this point I would like to avoid inverse problems. This I think is 
a perfect subject for future work, and will be mentioned in the outlook  \atline{633-635}} 

Specific comments 

{\it 
To assess the quality of the response surface, I think it'll be more 
appropriate to use "holdout" model runs, and compare the accuracy of the 
response surface to these model runs. All of the figures 4-9 seem to 
be using the training data to show the quality of the PC-based response 
surface. Maybe you could use something like a collection of 100 model 
runs at inputs chosen uniformly over the input space for this. 
}

{\bf My response: }{\color{blue} In the design example of section 5 I use a hold out model of $N_{rs} = 100$ uniform samples over the model parameter domain. I mention in 4.6 (\atline{497-499}) that a random sample of  $N_{rs} = 100$ was used to verify the accuracy of the response surface.} 

{\it 
The "extrapolation" points are difficult for me to understand. Are these 
just prediction of the code response? Are they extrapolations because 
the inputs are far away from the training set? [I doubt this because 
polynomials would be crummy for this purpose]. If they're just "prediction" 
points used for a holdout test of the response surface? If so, I think 
the term "extrapolation" is unwarrented. 
}
{\bf My response: }{\color{blue} I agree the "extrapolation" could imply points outside of the domain of $\mathbf{\lambda}$. The title of 4.7 is changed to "Predictions"} 

{\it 
The work of Lee et al. (2006) seems to be dealing with estimation 
and uncertainties in a charged particle accelerator. How does this 
relate to this paper? Some mention of the distinction should be 
given. 
Lee, Herbert KH, et al. "Inferring particle distribution in a 
proton accelerator experiment." Bayesian Analysis 1.2 (2006): 249-264. 
}

{\bf My response: }{\color{blue} Lee et al. treating a classical inverse problem, by using an emulator $\mathcal{M}$, measurements (projections of the distribution) and 
infer initial conditions (initial distributions).  While, in principe, with the UQ Ansatz inverse problems can be covered, I will not yet open this box. I will distinct Lee et al in the introduction \atline{53-62}.} 


{\it 
For Fig 12, are the 95\% uncertainties of the response surface are computed 
using an assumption of independent, identically distributed errors for 
the difference between the response surface surrogate and the actual 
code? I suspect so, but this is not an appropriate assumption - the 
errors are going to be very structured as a function of xi. I'd be tempted 
to go without this plot since it is trying to advertise something that I 
don't think is correct. Standard regression uncertainties all assume 
iid errors which are not the case here when one is using a polynomial surface 
to approximate a smooth model response. 
}

{\bf My response: }{\color{blue} From my surrogate model I can calculate the 95 CL.
In Fig 12 I simply show that predictions red points (before I called them Extrapolation) are within 95 CL. The prediction are based on iid of the model parameters over the 
full domain. 

I think I do not understand what you mean with $\xi$ are very structured and what the implication is.

At the moment I would like to stay with that figure.
} 



{\bf Referee \#3 (Remarks to the Author): }

{\it The paper reflects some misunderstanding concerning fundamental probabilistic concepts. For instance, on one 129, reference is made to stochastic processes, when in reality these expansions are for random variables. }

{\bf My response: }{\color{blue} I corrected this mistake and use the proper term random variable  \atline{137}.}


{\it Also, on line 150-155, it is stated that intrusive methods are P times more computationally expensive than the non-intrusive methods. This is also incorrect. The P equations in intrusive methods are coupled. Non-intrusive methods involve solving deterministic equations a large number of times. The comparison, as given in the paper, is thus completely baseless.}

{\bf My response: }{\color{blue}  I will only mention the need for modification and drop the statements w.r.t. explicit computational complexity, see  \atline{169-170} }

{\it The procedure, as described in the paper has been automated and described in many other publications. The insight provided following the computations is lacking and hindered by lack of sufficient perspective on probabilistic models. }

{\bf My response: }{\color{blue} The aim of this paper is to apply known methods/procedures to the field of charged particle
accelerators. I significantly modified the paper along these lines.












 \end{document}
 
