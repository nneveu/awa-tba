\documentclass{article}
\usepackage{latexsym}
\usepackage{color}
\setlength{\parskip}{3mm}
\setlength{\parindent}{0pt}
\usepackage[margin=1in]{geometry}

\newcommand{\TODO}[1]{\color{green}\noindent[{\bf TODO: #1}]\color{black}}
\newcommand{\atline}[1]{\color{red} Lines: #1\color{blue}}
\begin{document}
\title {Paper  \#ZW10157}
\author {N. Neveu, et. al}

\maketitle
Dear reviewers and editors,

Thank you for considering the manuscript.
We appreciate the constructive criticism. 
It helped improve the quality of this manuscript.

We have carefully answered all your questions and implemented your suggestions to the best of our knowledge.
The details can be found in the following pages.

With the best regards  \\[3mm]
 
Nicole Neveu, et. al

\pagebreak

{\bf Problems with the manuscript:}

{ Figure(s) [10,11] 
	Please remove background grid and/or shading from your figures.

* Please provide a figure file (.ps, .eps, .pdf, .jpg, or .png format)
that does not require tikz commands in the text source file to
display properly. Figure files should be self-contained and we do
not support the tikz format in the production process. We will also
need a revised text source file.}

{\bf Our response:} {\color{blue} The grid has been removed from Figures 10 and 11.
PDF versions of the tikz pictures have been made. All updated figures are provided 
along with an updated source file.}


{\bf REFEREE A (Remarks to Author)}

1) Remove the sections of code from the text. This adds no value to
	the paper and should be left for a User’s Manual.

{\bf Our response:} {\color{blue} The code examples in sections IV-V have been removed.}


2) Provide a stronger physics basis for the AWA optimization. Why is
this problem of beam squeeze challenging enough to warrant solution
with such a massively capable solver? Could the same solution have
been arrived at with something simpler? Describe the important
component of physics that come into play in the beam squeeze, whatever
they are – space charge, etc. After all, this is primarily a physics
journal so the physics context for the application needs to be clear.

{\bf Our response:} {\color{blue} XXXXX.}



{\bf REFEREE B (Remarks to Author)}

First, optimization systems based on PISA have been in use in
accelerator design since 2005, and NSGA-II’s use dates from that time
too. References to early PISA-based systems and NSGAII applications
are missing from the authors’ reference list, and the authors must
correct this. Three groups of references are provided to assist in
addressing these omissions: 1) early PISAbased systems; 2) a small
sample of NSGA-II applications and implementations (there are MANY
more as the NSGA-II algorithm is easy to understand and fairly
straight forward to implement); and 3) a survey of accelerator physics
and genetic and evolutionary application usage as of 2013.

1) References for 4 PISA-based optimization systems:

I.V. Bazarov and C.K. Sinclair. "Multivariate optimization of a high
brightness dc gun photoinjector," Phys. Rev. ST Accel. Beams 8,
034202, 2005

Y.C. Chao et al. "TRIUMF-VECC Electron Linac Beam Dynamics
Optimization," ICAP09, San Francisco, 2009, THPSC012

A. Hofler et al. "Optimizing SRF Gun Cavity Profiles in a Genetic
Algorithm Framework," ICAP09, San Francisco, 2009, THPSC020

M. Marchetto et al. "Beam Dynamics Optimization of the TRIUMF ELINAC
Injector," PAC09, Vancouver, BC, Canada, 2009, TH6PFP097

A. Hofler and P. Evtushenko. “Optimizing RF Gun Cavity Geometry within
an Automated Injector Design System,” NAPAC11, New York, 2011, TUODS6

I.V. Bazarov et al., "Comparison of dc and superconducting rf
photoemission guns for high brightness high average current beam
production", Phys. Rev. ST Accel. Beams 14, 072001, 2011

A. Hofler, “Optimization Framework for a Radio Frequency Gun Based
Injector,” Ph.D. thesis, Old Dominion University, Norfolk, Virginia,
USA, 2012

C. Gong and Y.C. Chao, "The TRIUMF Optimization Platform and
Application to the E-LINAC Injector", ICAP2012, Rostock-Warnemunde,
Germany, 2012, TUABI1

A. Hofler et al., "Innovative applications of genetic algorithms to
problems in accelerator physics", Phys. Rev. ST Accel. Beams 16,
010101, 2013

B. Terzic et al., "Simultaneous optimization of the cavity heat load
and trip rates in linacs using a genetic algorithm", Phys. Rev. ST
Accel. Beams 17, 101003, 2014

C. Gong, A Novel Optimization Platform and Its Applications to the
TRIUMF Energy Recovery Linac, Ph.D. thesis, The University of
Columbia, October 2015

C. Gong and Y.C. Chao, "The TRIUMF Optimization Platform and
Application to the E-LINAC Injector", ICAP2015, Shanghai, China, 2015,
THCWC4

C. Gulliford et al., "Multiobjective optimizations of a novel
cryocooled dc gun based ultrafast electron diffraction beam line",
Phys. Rev. Accel. Beams 19, 093402, 2016

C. Gulliford et al., "Multiobjective optimization design of an rf gun
based electron diffraction beam line", Phys. Rev. Accel. Beams 20,
033401, 2017

2) Optimization systems using NSGA-II (a few example references):

L. Emery. PAC 2005, Knoxville, TN, USA, May 2005, RPPP047

M. Borland et al. PAC 2009, Vancouver, BC, Canada, May 2009, TH6PFP062

Pulsar Physics. General Particle Tracer, http://www.pulsar.nl/gpt

3) Survey with various references showing GA/EA usage in Accelerator
Physics: A. Hofler. “Genetic Algorithms and Their Applications in
Accelerator Physics,” NAPAC2013, Pasadena, CA, USA, 2013, THTB1

{\bf Our response:} {\color{blue} We agree that some references of import were missing and should have been included.
Especially, the following: 

I.V. Bazarov and C.K. Sinclair. "Multivariate optimization of a high
brightness dc gun photoinjector," Phys. Rev. ST Accel. Beams 8,
034202, 2005

A. Hofler et al., "Innovative applications of genetic algorithms to
problems in accelerator physics", Phys. Rev. ST Accel. Beams 16,
010101, 2013


}





Second, due to the thoroughness and success of Bazarov and Sinclair’s
optimization (PRSTAB 2005 provided above), genetic and evolutionary
algorithm-based optimizations are standard practice in accelerator
design, and researchers are now looking at ways to use machine
learning and AI techniques in accelerator design and optimization.
Manual parameter scans, as mentioned in Section I lines 17-19, are
useful, but they are by no means the sole optimization technique in
use today. The statement made in lines 17-19 was true prior to 2005
but not now. The authors should consider providing a different and
more compelling motivator to encourage users to consider the system
they have developed.

{\bf Our response:} {\color{blue} XXXXX.}

Third, several of the features of this system are available is
existing systems but are not acknowledged:

- parallel/distributed computation: the survey article mentioned above
provides references to several examples

- Section II lines 115-125 blurring generations: APISA and General
Particle Tracer’s optimization system start evaluating the next
generation before all individuals in the present generation complete,
too.

- Section III subsection D lines 229-231 Geometry optimization. With
PISA-based systems, Bazarov and Hofler have done this (see PISA-based
reference list). Others have performed geometry optimization with
different GA/EA based systems (again the survey reference mentions
several.)

{\bf Our response:} {\color{blue} XXXXX.}

That said, a general-purpose optimization system is a useful addition
to the design tools for accelerators. The way to entice prospective
users through PRAB is to introduce the system with a very brief
general overview and then focus on the results obtained using the
system. The detailed discussion of the design and implementation with
code snippets should be provided as Supplemental Material. PRAB is not
a computer journal. Salient implementation or design points if
necessary can be summarized in short Appendices; the main discussion
of the paper should be the optimization results not the details of the
implementation and design. For example, in the main text, it is
reasonable to say that the optimization system provides a flexible and
extensible meta-language for setting up optimization problems (covered
in detail in Section III subsection E) and then direct the reader to
an Appendix or the Supplemental Material for the details provided in
Section III subsection E. Bear in mind, though, that several of the
PISA-based systems mentioned above, also, provide expression
capabilities and should be referenced. The one presented here is more
flexible and has the advantage of being easily extended. Both are
distinguishing features. The authors need to emphasize what
differentiates their system from 13+ years of development in the area
of GA/EA-based optimization in accelerator design and then demonstrate
that the results produced are exemplary or could not have been
produced otherwise. Or they should focus on how their system advances
the application of the target AWA machine (the Gulliford references
use this approach).

Following are comments/suggestions/questions organized by paper
section.

Abstract:

In this paper, we present the implementation of such a general-purpose
framework for simulation-based multi-objective optimization methods
that allows the automatic investigation of optimal sets of machine
parameters. The implementation is based on a master/slave paradigm,
employing several masters that govern a set of slaves executing
simulations and performing optimization tasks. -> In this paper, we
present a general-purpose framework for simulation-based
multi-objective optimization methods that allows the automatic
investigation of optimal sets of machine parameters.

The high charge beam line at the Argonne Wakefield Accelerator
Facility was used as the beam dynamics model. The 3D beam size,
transverse momentum, and energy spread were optimized. Please state
the significance of the optimization. Is the solution found better
than the original design? Then provide a comparison with the baseline
solution as Bazarov and Sinclair did in their 2005 paper. Is it the
first such solution for the AWA? Then make that clear. Has this
solution been used in the machine? Then provide a comparison between
prediction and operational experience.

Section I

line 35: trade of -$>$ trade off

{\bf Our response:} {\color{blue} This typo is corrected in the updated manuscript.}

Section II

lines 53-74. The car example is from Deb’s book but not cited. This
example can probably be removed as EA/GA methodology is established
practice now.

Section III

subsection A

Needs to be updated to reflect developments in accelerator physics.

subsections B-F

Replace with a brief overview and put the details in Supplemental
Material

Section VI: 
Code and input file details should be in Supplemental Material

Section V, subsection B:
Code and input file details should be in Supplemental Material

{\bf Our response:} {\color{blue} The code examples in sections VI-V has been removed from the updated manuscript. 
	A sentence and link to the repository have been provided. Interested parties can refer to the repository.}

subsubsection 3:
line 427 to to -$>$ to

{\bf Our response:} {\color{blue} This typo is corrected in the updated manuscript.}

subsubsection 4:
FIGS. 11/12 The discussed case in FIG. 12 is not on either front in
FIG. 11. Explain why a nonfront solution is used otherwise the
optimization exercise is moot.

{\bf Our response:} {\color{blue} This was a plotting mistake, and has been corrected in the updated Fig. 11.}

TABLE IV is provided and not discussed. Are these the settings for the
solution in FIG. 12? The table’s title is confusing. If the table
contents are input parameters for an optimization, shouldn’t they have
ranges of values? If the values are a selected solution, shouldn’t the
title state that directly?

{\bf Our response:} {\color{blue} This table shows input parameters for one optimized solution (the solution in Fig. 12). 
The table title has been changed to the following: ``Input Parameters for the optimized solution shown in Figures~11 and 12."}


 \end{document}
 
