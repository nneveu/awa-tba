\documentclass{article}
\usepackage{latexsym}
\usepackage{color}
\setlength{\parskip}{3mm}
\setlength{\parindent}{0pt}
\usepackage[margin=1in]{geometry}

\newcommand{\TODO}[1]{\color{green}\noindent[{\bf TODO: #1}]\color{black}}
\newcommand{\atline}[1]{\color{red} Lines: #1\color{blue}}
\begin{document}
\title {Paper  \#ZW10157}
\author {N. Neveu, et. al}

\maketitle
Dear reviewer and editors,

Thank you for the additional comments.
We have moved all code implementation details to appendices.
In addition, further revisions have been made as suggested.
We hope to show we value the constructive criticism provided.
The details can be found in the following pages.

With the best regards,  \\[3mm]
 
Nicole Neveu, et. al

\pagebreak

{\bf REFEREE B (Remarks to Author), Version 2}


My thanks, also, to the authors for updating the cited references.
When blocks of references are provided, conventionally, they are
listed chronologically to properly recognize the contributions of
pioneers, and the authors should reorder their citations to follow
this convention. 

{\bf Our response:} {\color{blue} The references have been 
	reordered chronologically, to recognize contribution order.}

For the Evolutionary Algorithm/Genetic Algorithm
(EA/GA)-based systems used in accelerator physics, I would ask them to
reconsider their choice to omit references by C. Gong and Y.C. Chao as
that optimization package allows users to define problems that use
multiple simulation programs in parallel and in series with various
interdependencies and execution times and manages all the complexity
for the user. Based on my understanding of the authors' system, this
is a common feature of both. The goal of publication is to introduce
new work to the community and to also recognize similar contributions
of others for comparison purposes.

{\bf Our response:} {\color{blue} The following references provided in the 
	referees first comments were added to the updated manuscript: 

Y.C. Chao et al. "TRIUMF-VECC Electron Linac Beam Dynamics
Optimization," ICAP09, San Francisco, 2009, THPSC012

C. Gong and Y.C. Chao, "The TRIUMF Optimization Platform and
Application to the E-LINAC Injector", ICAP2012, Rostock-Warnemunde,
Germany, 2012, TUABI1

C. Gong, A Novel Optimization Platform and Its Applications to the
TRIUMF Energy Recovery Linac, Ph.D. thesis, The University of
Columbia, October 2015

C. Gong and Y.C. Chao, "The TRIUMF Optimization Platform and
Application to the E-LINAC Injector", ICAP2015, Shanghai, China, 2015,
THCWC4}


I recognize the authors moved some of the source code listings to a
GitHub repository (may not comply with PRAB publishing practices as
PRAB provides Supplemental Materials storage and access on its web
site for each paper providing Supplemental Materials). 

{\bf Our response:} {\color{blue} The github link was removed,
	and Appendix B was added to provide an example OPAL 
	optimization input file.}



I stand by my initial assessment and believe more can be summarized with details
moved to Appendices or Supplemental Materials without diminishing the
authors' accomplishments. It is true that ``computing and algorithms''
are listed under PRAB's scope, but keeping the quite detailed
explanations of the implementation and software listings risks losing
readers interested in optimization systems in general. It would
behoove the authors to note that both reviewers commented on this in
different ways and that the recommendations were made with the goal of
increasing readership and readability. The authors' choice to ignore
the advice is shortsighted as it is possible to describe the
complexity and novel features of the system to readers who are and are
not adept computer programmers.

As an aside to the authors, I believe ``computing and algorithms''
were originally included in PRAB's scope to allow publications
discussing novel or new software requirements or
implementation details as they relate to applications specific to the
accelerator physics community such as field solvers, beam dynamics
codes, beam diagnostic data analysis, distributed control system
designs and communication protocols, etc. Some examples that do so
without detailed software implementation discussions include for
distributed processing for EA/GAs Bazarov and Sinclair's 2005 paper
and for methods for simulating coherent synchrotron radiation
Borland's 2001 and Agoh and Yokoya's 2004 papers. The ``rumor
network'' for message passing the authors employed may fall in the
category of new or novel for its use in accelerator physics
applications, but master/slave, C$++$ classes, inheritance, and
functors do not. As suggested in my initial review, much of the latter
can be summarized with the details moved to an Appendix or
Supplemental Materials. Readers who are truly interested in that level
of detail will go to the Appendices or Supplemental Materials. I
recommend to the authors a question that managers often have to ask
themselves: Is it more important to be right or effective? Do you want
to be right on a technicality about ``computing and algorithms,'' or
do you want to be effective in reaching as many readers as possible?


{\bf Our response:} {\color{blue} }




Sundry proofreading corrections listed by section.
I. paragraph 3: NSGA-II is mentioned but not cited.

{\bf Our response:} {\color{blue} The citation has been added.}


II. A. paragraph 4: By calling the selector --$>$ Calling the selector

{\bf Our response:} {\color{blue} This is corrected in the updated manuscript.}

paragraph 4
also present in GPT's [ref] optimization system --> also
present in Cornell's [2005 PRSTAB] and GPT's [ref] optimization systems
(Note to the authors: Cornell's implementation in APISA pre-dates ALL
of GPT's work with EA/GA's.)

{\bf Our response:} {\color{blue} The reference to Cornell was added.}

III. A: paragraph 2
search algorithm to get stuck --> search algorithm getting
stuck

{\bf Our response:} {\color{blue} This grammar error is corrected in the updated manuscript.}

III. B. paragraph 3
checking for idle WORKER --> checking for idle WORKERs

{\bf Our response:} {\color{blue} This typo is corrected in the updated manuscript.}

Fig. 3 caption
$r_k$ --> $r_1$
The work $W_j$ are beam dynamics simulation within OPAL.
--> $W_j$ performs beam dynamics simulations within OPAL.

{\bf Our response:} {\color{blue} The figure caption has been corrected.}

III. C. paragraph 1
shown in Listing 2 --> shown in Listing 1

III. D. paragraph 1
The author's use of ``approach'' is ambiguous because it could be
interpreted as applying EA/GA multi-objective optimization to cavity
geometries since the emphasis of the paper is on providing a flexible
interface to EA/GAs or approach could mean the framework's software
design which is what, I think, they meant to convey. The authors
should make clear which ``approach'' they want the reader to consider.
A suggested change is:

general nature of our approach --> general nature of our
framework's software design

{\bf Our response:} {\color{blue} This suggestion was used on line xx.}


paragraph 1: the API in Listing 3 --$>$ the API in Listing 2

paragraph 1: for those three --$>$ for three

{\bf Our response:} {\color{blue} This was corrected in the updated manuscript.}

III. E. paragraph 1:
An expression parser --> an expression parser
(Note to authors: ``an expression parser'' is a continuation of the
sentence started before the code example introduced as an example.)

paragraph 3
As shown in Listing 5 --> As shown in Listing 3

IV. paragraph 1:
in Listing 3 --> in Listing 1

V. B. 2 paragraph 2:
, and the to prevent --> and to prevent

Table III description, paragraph 2, and paragraph 3
twenty four --> twenty-four

paragraph 3
but from Fig. 9, but this is --> but from Fig. 9, this is

paragraph 3
15 less --> 15 fewer

V. B. 3
paragraph 2 (reported as corrected but is not in the version provided
to reviewers)
to to --> to

V. B. 4
Fig. 11 Vertical axis units are missing.




{\bf REFEREE B (Remarks to Author), Version 1}

\vspace{1em}
That said, a general-purpose optimization system is a useful addition
to the design tools for accelerators. The way to entice prospective
users through PRAB is to introduce the system with a very brief
general overview and then focus on the results obtained using the
system. The detailed discussion of the design and implementation with
code snippets should be provided as Supplemental Material. PRAB is not
a computer journal. Salient implementation or design points if
necessary can be summarized in short Appendices; the main discussion
of the paper should be the optimization results not the details of the
implementation and design. 

subsections B-F:
Replace with a brief overview and put the details in Supplemental
Material

{\bf Our response: } {\color{blue} This section has been shortened. However we would like to 
leave the existing description, in order to drive home the fact that the framework can be 
easily extended with other optimization algorithms.}

Section VI: 
Code and input file details should be in Supplemental Material

Section V, subsection B:
Code and input file details should be in Supplemental Material

{\bf Our response:} {\color{blue} The code examples in sections VI-V have been removed from the updated manuscript. 
	A sentence and link to the repository have been provided. Interested parties can refer to the repository.}



 \end{document}
 





